% Options for packages loaded elsewhere
\PassOptionsToPackage{unicode}{hyperref}
\PassOptionsToPackage{hyphens}{url}
\PassOptionsToPackage{dvipsnames,svgnames,x11names}{xcolor}
%
\documentclass[
]{article}
\usepackage{amsmath,amssymb}
\usepackage{iftex}
\ifPDFTeX
  \usepackage[T1]{fontenc}
  \usepackage[utf8]{inputenc}
  \usepackage{textcomp} % provide euro and other symbols
\else % if luatex or xetex
  \usepackage{unicode-math} % this also loads fontspec
  \defaultfontfeatures{Scale=MatchLowercase}
  \defaultfontfeatures[\rmfamily]{Ligatures=TeX,Scale=1}
\fi
\usepackage{lmodern}
\ifPDFTeX\else
  % xetex/luatex font selection
\fi
% Use upquote if available, for straight quotes in verbatim environments
\IfFileExists{upquote.sty}{\usepackage{upquote}}{}
\IfFileExists{microtype.sty}{% use microtype if available
  \usepackage[]{microtype}
  \UseMicrotypeSet[protrusion]{basicmath} % disable protrusion for tt fonts
}{}
\makeatletter
\@ifundefined{KOMAClassName}{% if non-KOMA class
  \IfFileExists{parskip.sty}{%
    \usepackage{parskip}
  }{% else
    \setlength{\parindent}{0pt}
    \setlength{\parskip}{6pt plus 2pt minus 1pt}}
}{% if KOMA class
  \KOMAoptions{parskip=half}}
\makeatother
\usepackage{xcolor}
\usepackage[margin=1in]{geometry}
\usepackage{color}
\usepackage{fancyvrb}
\newcommand{\VerbBar}{|}
\newcommand{\VERB}{\Verb[commandchars=\\\{\}]}
\DefineVerbatimEnvironment{Highlighting}{Verbatim}{commandchars=\\\{\}}
% Add ',fontsize=\small' for more characters per line
\usepackage{framed}
\definecolor{shadecolor}{RGB}{248,248,248}
\newenvironment{Shaded}{\begin{snugshade}}{\end{snugshade}}
\newcommand{\AlertTok}[1]{\textcolor[rgb]{0.94,0.16,0.16}{#1}}
\newcommand{\AnnotationTok}[1]{\textcolor[rgb]{0.56,0.35,0.01}{\textbf{\textit{#1}}}}
\newcommand{\AttributeTok}[1]{\textcolor[rgb]{0.13,0.29,0.53}{#1}}
\newcommand{\BaseNTok}[1]{\textcolor[rgb]{0.00,0.00,0.81}{#1}}
\newcommand{\BuiltInTok}[1]{#1}
\newcommand{\CharTok}[1]{\textcolor[rgb]{0.31,0.60,0.02}{#1}}
\newcommand{\CommentTok}[1]{\textcolor[rgb]{0.56,0.35,0.01}{\textit{#1}}}
\newcommand{\CommentVarTok}[1]{\textcolor[rgb]{0.56,0.35,0.01}{\textbf{\textit{#1}}}}
\newcommand{\ConstantTok}[1]{\textcolor[rgb]{0.56,0.35,0.01}{#1}}
\newcommand{\ControlFlowTok}[1]{\textcolor[rgb]{0.13,0.29,0.53}{\textbf{#1}}}
\newcommand{\DataTypeTok}[1]{\textcolor[rgb]{0.13,0.29,0.53}{#1}}
\newcommand{\DecValTok}[1]{\textcolor[rgb]{0.00,0.00,0.81}{#1}}
\newcommand{\DocumentationTok}[1]{\textcolor[rgb]{0.56,0.35,0.01}{\textbf{\textit{#1}}}}
\newcommand{\ErrorTok}[1]{\textcolor[rgb]{0.64,0.00,0.00}{\textbf{#1}}}
\newcommand{\ExtensionTok}[1]{#1}
\newcommand{\FloatTok}[1]{\textcolor[rgb]{0.00,0.00,0.81}{#1}}
\newcommand{\FunctionTok}[1]{\textcolor[rgb]{0.13,0.29,0.53}{\textbf{#1}}}
\newcommand{\ImportTok}[1]{#1}
\newcommand{\InformationTok}[1]{\textcolor[rgb]{0.56,0.35,0.01}{\textbf{\textit{#1}}}}
\newcommand{\KeywordTok}[1]{\textcolor[rgb]{0.13,0.29,0.53}{\textbf{#1}}}
\newcommand{\NormalTok}[1]{#1}
\newcommand{\OperatorTok}[1]{\textcolor[rgb]{0.81,0.36,0.00}{\textbf{#1}}}
\newcommand{\OtherTok}[1]{\textcolor[rgb]{0.56,0.35,0.01}{#1}}
\newcommand{\PreprocessorTok}[1]{\textcolor[rgb]{0.56,0.35,0.01}{\textit{#1}}}
\newcommand{\RegionMarkerTok}[1]{#1}
\newcommand{\SpecialCharTok}[1]{\textcolor[rgb]{0.81,0.36,0.00}{\textbf{#1}}}
\newcommand{\SpecialStringTok}[1]{\textcolor[rgb]{0.31,0.60,0.02}{#1}}
\newcommand{\StringTok}[1]{\textcolor[rgb]{0.31,0.60,0.02}{#1}}
\newcommand{\VariableTok}[1]{\textcolor[rgb]{0.00,0.00,0.00}{#1}}
\newcommand{\VerbatimStringTok}[1]{\textcolor[rgb]{0.31,0.60,0.02}{#1}}
\newcommand{\WarningTok}[1]{\textcolor[rgb]{0.56,0.35,0.01}{\textbf{\textit{#1}}}}
\usepackage{longtable,booktabs,array}
\usepackage{calc} % for calculating minipage widths
% Correct order of tables after \paragraph or \subparagraph
\usepackage{etoolbox}
\makeatletter
\patchcmd\longtable{\par}{\if@noskipsec\mbox{}\fi\par}{}{}
\makeatother
% Allow footnotes in longtable head/foot
\IfFileExists{footnotehyper.sty}{\usepackage{footnotehyper}}{\usepackage{footnote}}
\makesavenoteenv{longtable}
\usepackage{graphicx}
\makeatletter
\newsavebox\pandoc@box
\newcommand*\pandocbounded[1]{% scales image to fit in text height/width
  \sbox\pandoc@box{#1}%
  \Gscale@div\@tempa{\textheight}{\dimexpr\ht\pandoc@box+\dp\pandoc@box\relax}%
  \Gscale@div\@tempb{\linewidth}{\wd\pandoc@box}%
  \ifdim\@tempb\p@<\@tempa\p@\let\@tempa\@tempb\fi% select the smaller of both
  \ifdim\@tempa\p@<\p@\scalebox{\@tempa}{\usebox\pandoc@box}%
  \else\usebox{\pandoc@box}%
  \fi%
}
% Set default figure placement to htbp
\def\fps@figure{htbp}
\makeatother
\setlength{\emergencystretch}{3em} % prevent overfull lines
\providecommand{\tightlist}{%
  \setlength{\itemsep}{0pt}\setlength{\parskip}{0pt}}
\setcounter{secnumdepth}{-\maxdimen} % remove section numbering
% definitions for citeproc citations
\NewDocumentCommand\citeproctext{}{}
\NewDocumentCommand\citeproc{mm}{%
  \begingroup\def\citeproctext{#2}\cite{#1}\endgroup}
\makeatletter
 % allow citations to break across lines
 \let\@cite@ofmt\@firstofone
 % avoid brackets around text for \cite:
 \def\@biblabel#1{}
 \def\@cite#1#2{{#1\if@tempswa , #2\fi}}
\makeatother
\newlength{\cslhangindent}
\setlength{\cslhangindent}{1.5em}
\newlength{\csllabelwidth}
\setlength{\csllabelwidth}{3em}
\newenvironment{CSLReferences}[2] % #1 hanging-indent, #2 entry-spacing
 {\begin{list}{}{%
  \setlength{\itemindent}{0pt}
  \setlength{\leftmargin}{0pt}
  \setlength{\parsep}{0pt}
  % turn on hanging indent if param 1 is 1
  \ifodd #1
   \setlength{\leftmargin}{\cslhangindent}
   \setlength{\itemindent}{-1\cslhangindent}
  \fi
  % set entry spacing
  \setlength{\itemsep}{#2\baselineskip}}}
 {\end{list}}
\usepackage{calc}
\newcommand{\CSLBlock}[1]{\hfill\break\parbox[t]{\linewidth}{\strut\ignorespaces#1\strut}}
\newcommand{\CSLLeftMargin}[1]{\parbox[t]{\csllabelwidth}{\strut#1\strut}}
\newcommand{\CSLRightInline}[1]{\parbox[t]{\linewidth - \csllabelwidth}{\strut#1\strut}}
\newcommand{\CSLIndent}[1]{\hspace{\cslhangindent}#1}
\usepackage{fancyhdr}
\pagestyle{fancy}
\fancyhf{}
\lfoot[\thepage]{}
\rfoot[]{\thepage}
\fontsize{12}{22}
\selectfont
\usepackage{booktabs}
\usepackage{longtable}
\usepackage{array}
\usepackage{multirow}
\usepackage{wrapfig}
\usepackage{float}
\usepackage{colortbl}
\usepackage{pdflscape}
\usepackage{tabu}
\usepackage{threeparttable}
\usepackage{threeparttablex}
\usepackage[normalem]{ulem}
\usepackage{makecell}
\usepackage{xcolor}
\usepackage{bookmark}
\IfFileExists{xurl.sty}{\usepackage{xurl}}{} % add URL line breaks if available
\urlstyle{same}
\hypersetup{
  colorlinks=true,
  linkcolor={blue},
  filecolor={Maroon},
  citecolor={Blue},
  urlcolor={Blue},
  pdfcreator={LaTeX via pandoc}}

\title{\includegraphics[width=6cm,height=\textheight,keepaspectratio]{figures/IEO-logo.jpg}}
\author{}
\date{\vspace{-2.5em}}

\begin{document}
\maketitle


\pagenumbering{gobble}

%\begin{titlepage}
\begin{flushleft}
\Large{\textbf{Material Suplementario 1}}\\
\vspace*{2\baselineskip}
\LARGE{\textbf{Análisis del viento de Levante y su impacto en reclutamiento de coquina \textit{Donax trunculus} durante la última década}}\\
\vspace*{5\baselineskip}
\Large{Project FEMP 04}\\
\vspace*{1\baselineskip}
\Large{Instituto Español de Oceanografía, Cádiz }\\
\vspace*{4\baselineskip}
\end{flushleft}
\begin{flushright}
\large{\textit{Mauricio Mardones}}\\
\large{\textit{Marina Delgado}}\\
\large{\textit{Ricardo Sánchez}}\\
\vspace*{1\baselineskip}
\normalsize{\textbf{Fecha}}\\
August, 2025
\end{flushright}

% \end{titlepage}


\hypersetup{linkcolor = black}
\newpage
\pagenumbering{roman}
%\tableofcontents
%\addcontentsline{toc}{section}{\contentsname}

\newpage



\pagenumbering{arabic}
\hypersetup{linkcolor = blue}

{
\hypersetup{linkcolor=}
\setcounter{tocdepth}{3}
\tableofcontents
}
\pagebreak

\section{Contexto}\label{contexto}

\subsubsection{\texorpdfstring{Análisis de los vientos de Levante y su relación con el reclutamiento de la coquina (\emph{Donax trunculus})}{Análisis de los vientos de Levante y su relación con el reclutamiento de la coquina (Donax trunculus)}}\label{anuxe1lisis-de-los-vientos-de-levante-y-su-relaciuxf3n-con-el-reclutamiento-de-la-coquina-donax-trunculus}

Este estudio realiza un análisis detallado de los patrones de viento de Levante y Poniente a partir de datos meteorológicos recopilados entre 2013 y 2025. Se caracterizan aspectos clave como la frecuencia, duración e intensidad de estos eventos atmosféricos, con el objetivo de evaluar su posible influencia sobre variables poblacionales de la coquina \emph{Donax trunculus}, una especie bivalva de importancia ecológica y pesquera en el litoral suratlántico español.

\subsubsection{Objetivo}\label{objetivo}

El objetivo principal de este trabajo es identificar y cuantificar los patrones de viento dominantes y analizar su relación con parámetros poblacionales de \emph{Donax trunculus}. Para ello, se emplean datos de monitoreo biológico y pesquero obtenidos por el Instituto Español de Oceanografía (IEO-CSIC) en el marco del proyecto FEMP 04, con énfasis en las dinámicas de reclutamiento observadas en el Golfo de Cádiz.

\newpage

\section{Metodología}\label{metodologuxeda}

\begin{Shaded}
\begin{Highlighting}[]
\NormalTok{paquetes }\OtherTok{\textless{}{-}} \FunctionTok{c}\NormalTok{(}
  \StringTok{"readr"}\NormalTok{, }\StringTok{"dplyr"}\NormalTok{, }\StringTok{"lubridate"}\NormalTok{, }\StringTok{"stringr"}\NormalTok{, }\StringTok{"purrr"}\NormalTok{,}
  \StringTok{"ggplot2"}\NormalTok{, }\StringTok{"tidyr"}\NormalTok{, }\StringTok{"gridExtra"}\NormalTok{, }\StringTok{"viridis"}\NormalTok{, }\StringTok{"scales"}\NormalTok{,}
  \StringTok{"formatR"}\NormalTok{, }\StringTok{"ggpubr"}\NormalTok{, }\StringTok{"ggthemes"}\NormalTok{, }\StringTok{"kableExtra"}\NormalTok{, }\StringTok{"sjPlot"}\NormalTok{, }
  \StringTok{"broom"}\NormalTok{, }\StringTok{"kableExtra"}
\NormalTok{  )}

\NormalTok{purrr}\SpecialCharTok{::}\FunctionTok{walk}\NormalTok{(paquetes, library, }\AttributeTok{character.only =} \ConstantTok{TRUE}\NormalTok{)}
\end{Highlighting}
\end{Shaded}

\subsubsection{Datos de viento}\label{datos-de-viento}

El acceso al servicio de descargas de \textbf{Puertos del Estado} se hace desde la página web de oceanografía de \href{http://www.puertos.es/es-es/oceanografia/Paginas/portus.aspx}{Puertos del Estado}

Selección de estaciones y variables

\begin{itemize}
\tightlist
\item
  En el sistema \textbf{DescargaPortus}, se suele seleccionar entre varias estaciones meteorológicas costeras.
\item
  Elegimos \textbf{cuatro puntos} ubicados frente a la costa del Parque de Doñana en la provincia de Cádiz (Figure \ref{fig:mapa}).
\item
  Seleccionamos datos de \textbf{viento}, en particular \textbf{velocidad} y \textbf{dirección}.
\end{itemize}

\begin{figure}

{\centering \includegraphics[width=0.95\linewidth]{figures/mapa} 

}

\caption{Puntos seleccionados con la variable viento desde Puertos del Estado}\label{fig:mapa}
\end{figure}

Período y frecuencia de datos

\begin{itemize}
\tightlist
\item
  Un \textbf{rango temporal} (por ejemplo 2013‑2025).
\item
  Desechar valores nulos (ej. ‑999.9) siguiendo las recomendaciones de limpieza del manual.
\end{itemize}

Formato de descarga

\begin{itemize}
\tightlist
\item
  \texttt{.csv}en formato tabular con columnas como:
\item
  Fecha (GMT)
\item
  Velocidad del viento (m/s)
\item
  Dirección del viento (grados meteorológicos)
\end{itemize}

Calidad de datos

\begin{itemize}
\tightlist
\item
  Aplicamos filtros para descartar datos erróneos, como aquellos con velocidad negativa o direcciones fuera del rango 0--360º.
\end{itemize}

Procesamiento en R

\begin{itemize}
\tightlist
\item
  Finalmente, se importan en R usando funciones tipo \texttt{readr}, limpieza con \texttt{dplyr}, conversión de fechas con \texttt{lubridate}, etc.
\end{itemize}

Flujo de trabajo

\begin{longtable}[]{@{}
  >{\raggedright\arraybackslash}p{(\linewidth - 2\tabcolsep) * \real{0.3375}}
  >{\raggedright\arraybackslash}p{(\linewidth - 2\tabcolsep) * \real{0.6625}}@{}}
\toprule\noalign{}
\begin{minipage}[b]{\linewidth}\raggedright
Etapa
\end{minipage} & \begin{minipage}[b]{\linewidth}\raggedright
Acción realizada
\end{minipage} \\
\midrule\noalign{}
\endhead
\bottomrule\noalign{}
\endlastfoot
Selección de estaciones & Cuatro puntos frente a Doñana (Cádiz) \\
Variables & Viento: velocidad y dirección \\
Periodo de interés & Desde 2013 hasta el presente \\
Formato de descarga & CSV / delimitado con primer registro meta \\
Limpieza de datos & Eliminación de ``‑999.9'' y NA \\
Conversión de formatos & Fecha GMT a \texttt{POSIXct} o \texttt{Date} \\
\end{longtable}

\begin{Shaded}
\begin{Highlighting}[]
\CommentTok{\#Funcion para leer todos los archivos.}
\NormalTok{leer\_archivo\_viento }\OtherTok{\textless{}{-}} \ControlFlowTok{function}\NormalTok{(archivo) \{}
\NormalTok{  readr}\SpecialCharTok{::}\FunctionTok{read\_tsv}\NormalTok{(archivo, }\AttributeTok{skip =} \DecValTok{1}\NormalTok{,}
                  \AttributeTok{col\_names =} \FunctionTok{c}\NormalTok{(}\StringTok{"fecha\_raw"}\NormalTok{, }
                                \StringTok{"velocidad\_viento"}\NormalTok{, }
                                \StringTok{"direccion\_grados"}\NormalTok{),}
                  \AttributeTok{show\_col\_types =} \ConstantTok{FALSE}\NormalTok{) }\SpecialCharTok{\%\textgreater{}\%}
    \FunctionTok{mutate}\NormalTok{(}
      \AttributeTok{fecha\_raw =}\NormalTok{ stringr}\SpecialCharTok{::}\FunctionTok{str\_trim}\NormalTok{(fecha\_raw),}
      \AttributeTok{fecha =}\NormalTok{ lubridate}\SpecialCharTok{::}\FunctionTok{parse\_date\_time}\NormalTok{(fecha\_raw, }\AttributeTok{orders =} \StringTok{"Y m d H"}\NormalTok{, }\AttributeTok{tz =} \StringTok{"UTC"}\NormalTok{),}
      \AttributeTok{velocidad\_viento =} \FunctionTok{as.numeric}\NormalTok{(velocidad\_viento),}
      \AttributeTok{direccion\_grados =} \FunctionTok{as.numeric}\NormalTok{(direccion\_grados)}
\NormalTok{    ) }\SpecialCharTok{\%\textgreater{}\%}
\NormalTok{    dplyr}\SpecialCharTok{::}\FunctionTok{select}\NormalTok{(fecha, velocidad\_viento, direccion\_grados)}
\NormalTok{\}}

\NormalTok{datos\_viento }\OtherTok{\textless{}{-}}\NormalTok{ purrr}\SpecialCharTok{::}\FunctionTok{map\_dfr}\NormalTok{(rutas\_completas, leer\_archivo\_viento)}
\end{Highlighting}
\end{Shaded}

Los registros de viento fueron clasificados en tres categorías: \textbf{Levante}, \textbf{Poniente} u \textbf{Otro}, en base a la siguiente regla propuesta por Bartolomé López-Somoza (\citeproc{ref-bartolome1998viento}{1998}):

\[
\text{tipo\_viento} =
\begin{cases}
\text{"Levante"} & \text{si } 4 \leq V < 40 \text{ y } 67.5^\circ \leq D \leq 157.5^\circ \\
\text{"Poniente"} & \text{si } 8 \leq V < 40 \text{ y } 247.5^\circ \leq D \leq 292.5^\circ \\
\text{"Otro"} & \text{en cualquier otro caso}
\end{cases}
\]
Donde:

\begin{itemize}
\tightlist
\item
  \(V\) es la \textbf{velocidad del viento} en m/s.
\item
  \(D\) es la \textbf{dirección del viento} en grados meteorológicos.
\end{itemize}

Esta clasificación se implementó en R con \texttt{case\_when()} para etiquetar automáticamente cada observación horaria.

En terminos direccionales, la rosa de los vientos queda clasificada de la siguiente forma:

\begin{figure}

{\centering \includegraphics[width=1\linewidth]{Levante_Analisys_files/figure-latex/windrose-1} 

}

\caption{Rosa de vientos mostrando la clasificación de sectores para Levante y Poniente}\label{fig:windrose}
\end{figure}

\subsection{Datos de variables poblacionales de coquina}\label{datos-de-variables-poblacionales-de-coquina}

Utilizaremos el indice de reclutamiento que provien e de las tallas , calculado como la proporcion de individuos \texttt{\textless{}\ 15\ mm}.

\newpage

\section{Resultados}\label{resultados}

Analizamos la frecuencia de eventos de levante y poniente en la Figura \ref{fig:frequency-analysis2}.

\begin{figure}

{\centering \includegraphics[width=0.6\linewidth]{Levante_Analisys_files/figure-latex/frequency-analysis2-1} 

}

\caption{Distribución de velocidades del viento y frecuencia por dirección cardinal}\label{fig:frequency-analysis2}
\end{figure}

\newpage

También visualizamos la relación entre intensisdad y dias con levante y poniente en la Figura \ref{fig:intvel}

\begin{figure}

{\centering \includegraphics[width=0.6\linewidth]{Levante_Analisys_files/figure-latex/intvel-1} 

}

\caption{Relación entre intensisdad y dias con levante y poniente}\label{fig:intvel}
\end{figure}

Promedio por mes de los días con levante y poniente a través de los meses en la Figura \ref{fig:intvel2}.

\begin{figure}

{\centering \includegraphics[width=0.6\linewidth]{Levante_Analisys_files/figure-latex/intvel2-1} 

}

\caption{Relación entre intensisdad y dias con levante y poniente}\label{fig:intvel2}
\end{figure}
\newpage

Ahora por año y por mes

\begin{center}\includegraphics[width=1\linewidth]{Levante_Analisys_files/figure-latex/unnamed-chunk-3-1} \end{center}

\begin{center}\includegraphics[width=1\linewidth]{Levante_Analisys_files/figure-latex/unnamed-chunk-4-1} \end{center}

\begin{center}\includegraphics[width=1\linewidth]{Levante_Analisys_files/figure-latex/unnamed-chunk-5-1} \end{center}

\begin{center}\includegraphics[width=1\linewidth]{Levante_Analisys_files/figure-latex/unnamed-chunk-6-1} \end{center}
\newpage

\subsubsection{Series temporales del monitoreo poblacional de coquina}\label{series-temporales-del-monitoreo-poblacional-de-coquina}

\begin{center}\includegraphics[width=1\linewidth]{Levante_Analisys_files/figure-latex/unnamed-chunk-7-1} \end{center}

El plot muestra una marcada estacionalidad de los reclutamientos, con un pulso posterior a los meses estivales como lo indica Delgado \& Silva (\citeproc{ref-Delgado2018}{2018}) en la Figura \ref{fig:ciclo}.

\begin{figure}

{\centering \includegraphics[width=0.75\linewidth]{figures/ciclo} 

}

\caption{Schematic representation of the reproductive cycle, periods of emission of gametes and related recruitment events in populations of D. trunculus from SW Spain. Black symbols represent the C1 cohort (from February-March) and grey symbols represent the C2 cohort (from July)}\label{fig:ciclo}
\end{figure}
\newpage

\subsection{Modelos de correlacion entre Reclutamiento y Viento de Levante}\label{modelos-de-correlacion-entre-reclutamiento-y-viento-de-levante}

Uno las bases que estan en los objetos \texttt{serie\_levante} y \texttt{datos}

\begin{verbatim}
## Rows: 145
## Columns: 6
## $ año             <dbl> 2013, 2013, 2013, 2013, 2013, 2013, 2013, 2013, 2013, ~
## $ mes             <ord> Jan, Feb, Mar, Apr, May, Jun, Jul, Aug, Sep, Oct, Nov,~
## $ tipo_viento     <chr> "Levante", "Levante", "Levante", "Levante", "Levante",~
## $ dias_con_evento <int> 5, 4, 8, 4, 1, 1, 5, 2, 5, 6, 6, 15, 7, 1, 7, 4, 4, 3,~
## $ mes_num         <int> 1, 2, 3, 4, 5, 6, 7, 8, 9, 10, 11, 12, 1, 2, 3, 4, 5, ~
## $ año_mes         <date> 2013-01-01, 2013-02-01, 2013-03-01, 2013-04-01, 2013-~
\end{verbatim}

\begin{verbatim}
## Rows: 109
## Columns: 3
## Groups: ANO [12]
## $ ANO     <dbl> 2013, 2013, 2014, 2014, 2014, 2014, 2014, 2014, 2014, 2014, 20~
## $ mes_num <dbl> 10, 12, 2, 3, 4, 5, 6, 7, 8, 9, 10, 11, 12, 1, 2, 3, 4, 5, 6, ~
## $ meand15 <dbl> 0.0000000, 2.9239766, 12.7121216, 13.6718750, 37.3687894, 6.43~
\end{verbatim}

\begin{figure}

{\centering \includegraphics[width=0.75\linewidth]{Levante_Analisys_files/figure-latex/unnamed-chunk-10-1} 

}

\caption{Distribución de variablees objetivo para el análisis de correlación}\label{fig:unnamed-chunk-10-1}
\end{figure}
\begin{figure}

{\centering \includegraphics[width=0.75\linewidth]{Levante_Analisys_files/figure-latex/unnamed-chunk-10-2} 

}

\caption{Distribución de variablees objetivo para el análisis de correlación}\label{fig:unnamed-chunk-10-2}
\end{figure}

\begin{center}\includegraphics[width=1\linewidth]{Levante_Analisys_files/figure-latex/unnamed-chunk-11-1} \end{center}

\begin{table}[!h]
\centering
\caption{\label{tab:unnamed-chunk-13}Modelo lineal simple de reclutamiento de coquina y viento de levante}
\centering
\begin{tabular}[t]{l|r|r|r|r|r|r}
\hline
Término & Coeficiente & Error estándar & Estadístico t & Valor p & IC inferior 95\% & IC superior 95\%\\
\hline
\cellcolor{gray!10}{(Intercept)} & \cellcolor{gray!10}{24.657} & \cellcolor{gray!10}{2.644} & \cellcolor{gray!10}{9.323810} & \cellcolor{gray!10}{0.000} & \cellcolor{gray!10}{19.413} & \cellcolor{gray!10}{29.901}\\
\hline
dias\_con\_evento & -0.418 & 0.277 & -1.508633 & 0.134 & -0.967 & 0.131\\
\hline
\end{tabular}
\end{table}

En primera instancia, no se observa una correlación lineal significativa entre el viento de levante y el reclutamiento, con un coeficiente de determinación muy bajo (R2 = 0.021), indicando escaso poder explicativo.

A continuación, se procede a analizar la relación entre estas variables en una escala temporal diferente, específicamente a nivel semestral, evaluando la posible influencia del viento de levante sobre el reclutamiento de coquina.

\begin{center}\includegraphics[width=1\linewidth]{Levante_Analisys_files/figure-latex/unnamed-chunk-15-1} \end{center}

\begin{verbatim}
## [1] "\n=== CORRELACIÓN POR SEMESTRE ==="
\end{verbatim}

\begin{table}[!h]
\centering
\begin{tabular}[t]{r|r|r}
\hline
semestre & correlacion & n\_semestres\\
\hline
\cellcolor{gray!10}{1} & \cellcolor{gray!10}{-0.1966243} & \cellcolor{gray!10}{10}\\
\hline
2 & 0.5225889 & 10\\
\hline
\end{tabular}
\end{table}

\subsubsection{Modelos semestrales con interacción}\label{modelos-semestrales-con-interacciuxf3n}

\begin{table}[!h]
\centering
\begin{tabular}[t]{l|l|r|r|r}
\hline
  & Modelo & R2\_ajustado & AIC & p\_value\\
\hline
\cellcolor{gray!10}{Lineal} & \cellcolor{gray!10}{Lineal} & \cellcolor{gray!10}{-0.0554818} & \cellcolor{gray!10}{159.4193} & \cellcolor{gray!10}{0.9720995}\\
\hline
Cuadrático & Cuadrático & -0.0469758 & 160.1143 & 0.5739469\\
\hline
\cellcolor{gray!10}{Con semestre} & \cellcolor{gray!10}{Con semestre} & \cellcolor{gray!10}{0.3456430} & \cellcolor{gray!10}{150.7142} & \cellcolor{gray!10}{0.0105642}\\
\hline
Interacción & Interacción & 0.4262550 & 148.8723 & 0.0074818\\
\hline
\end{tabular}
\end{table}

Figura \ref{fig:resutlm}

\begin{center}\includegraphics[width=1\linewidth]{Levante_Analisys_files/figure-latex/resultlm-1} \end{center}

El gráfico muestra que la influencia del viento de Levante sobre el reclutamiento de coquina (\emph{Donax trunculus}) varía significativamente según el semestre del año, revelando una interacción estacional. Durante el primer semestre (enero a junio), que corresponde al periodo reproductivo, se observa un efecto negativo, es decir, a mayor número de días con viento de Levante, menor es el reclutamiento promedio. Esto sugiere que estos vientos podrían interferir con el éxito reproductivo, afectando la dispersión larval o la retención costera, o generando condiciones ambientales desfavorables en el hábitat de asentamiento, como cambios en temperatura, salinidad o estructura del sedimento. En cambio, en el segundo semestre (julio a diciembre), el efecto es positivo: un mayor número de días con viento se asocia con un mayor reclutamiento, lo que podría deberse a que estos vientos favorecen la acumulación de materia orgánica o sedimentos en las playas, mejorando las condiciones de crecimiento y refugio para los juveniles ya asentados. Esta inversión del efecto sugiere que el mismo factor ambiental puede tener consecuencias opuestas según el momento del ciclo de vida en que actúa, lo que subraya la importancia de considerar la estacionalidad en la evaluación del reclutamiento y en la gestión pesquera de esta especie.

\subsubsection{Trabajo futuro}\label{trabajo-futuro}

Incorporar otras variables predictoras como temperatura, clorofila, entre otras\ldots{}

\newpage

\section*{Referencias}\label{referencias}
\addcontentsline{toc}{section}{Referencias}

\phantomsection\label{refs}
\begin{CSLReferences}{1}{0}
\bibitem[\citeproctext]{ref-bartolome1998viento}
Bartolomé López-Somoza, E. (1998). \emph{El viento de levante en el puerto}. Ayuntamiento de El Puerto de Santa María.

\bibitem[\citeproctext]{ref-Delgado2018}
Delgado, M., \& Silva, L. (2018). {Timing variations and effects of size on the reproductive output of the wedge clam Donax trunculus (L. 1758) in the littoral of Huelva (SW Spain)}. \emph{Journal of the Marine Biological Association of the United Kingdom}, \emph{98}(2), 341--350. \url{https://doi.org/10.1017/S0025315416001429}

\end{CSLReferences}

\end{document}
