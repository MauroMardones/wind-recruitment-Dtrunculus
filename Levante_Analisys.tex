% Options for packages loaded elsewhere
\PassOptionsToPackage{unicode}{hyperref}
\PassOptionsToPackage{hyphens}{url}
\PassOptionsToPackage{dvipsnames,svgnames,x11names}{xcolor}
%
\documentclass[
]{article}
\usepackage{amsmath,amssymb}
\usepackage{iftex}
\ifPDFTeX
  \usepackage[T1]{fontenc}
  \usepackage[utf8]{inputenc}
  \usepackage{textcomp} % provide euro and other symbols
\else % if luatex or xetex
  \usepackage{unicode-math} % this also loads fontspec
  \defaultfontfeatures{Scale=MatchLowercase}
  \defaultfontfeatures[\rmfamily]{Ligatures=TeX,Scale=1}
\fi
\usepackage{lmodern}
\ifPDFTeX\else
  % xetex/luatex font selection
\fi
% Use upquote if available, for straight quotes in verbatim environments
\IfFileExists{upquote.sty}{\usepackage{upquote}}{}
\IfFileExists{microtype.sty}{% use microtype if available
  \usepackage[]{microtype}
  \UseMicrotypeSet[protrusion]{basicmath} % disable protrusion for tt fonts
}{}
\makeatletter
\@ifundefined{KOMAClassName}{% if non-KOMA class
  \IfFileExists{parskip.sty}{%
    \usepackage{parskip}
  }{% else
    \setlength{\parindent}{0pt}
    \setlength{\parskip}{6pt plus 2pt minus 1pt}}
}{% if KOMA class
  \KOMAoptions{parskip=half}}
\makeatother
\usepackage{xcolor}
\usepackage[margin=1in]{geometry}
\usepackage{color}
\usepackage{fancyvrb}
\newcommand{\VerbBar}{|}
\newcommand{\VERB}{\Verb[commandchars=\\\{\}]}
\DefineVerbatimEnvironment{Highlighting}{Verbatim}{commandchars=\\\{\}}
% Add ',fontsize=\small' for more characters per line
\usepackage{framed}
\definecolor{shadecolor}{RGB}{248,248,248}
\newenvironment{Shaded}{\begin{snugshade}}{\end{snugshade}}
\newcommand{\AlertTok}[1]{\textcolor[rgb]{0.94,0.16,0.16}{#1}}
\newcommand{\AnnotationTok}[1]{\textcolor[rgb]{0.56,0.35,0.01}{\textbf{\textit{#1}}}}
\newcommand{\AttributeTok}[1]{\textcolor[rgb]{0.13,0.29,0.53}{#1}}
\newcommand{\BaseNTok}[1]{\textcolor[rgb]{0.00,0.00,0.81}{#1}}
\newcommand{\BuiltInTok}[1]{#1}
\newcommand{\CharTok}[1]{\textcolor[rgb]{0.31,0.60,0.02}{#1}}
\newcommand{\CommentTok}[1]{\textcolor[rgb]{0.56,0.35,0.01}{\textit{#1}}}
\newcommand{\CommentVarTok}[1]{\textcolor[rgb]{0.56,0.35,0.01}{\textbf{\textit{#1}}}}
\newcommand{\ConstantTok}[1]{\textcolor[rgb]{0.56,0.35,0.01}{#1}}
\newcommand{\ControlFlowTok}[1]{\textcolor[rgb]{0.13,0.29,0.53}{\textbf{#1}}}
\newcommand{\DataTypeTok}[1]{\textcolor[rgb]{0.13,0.29,0.53}{#1}}
\newcommand{\DecValTok}[1]{\textcolor[rgb]{0.00,0.00,0.81}{#1}}
\newcommand{\DocumentationTok}[1]{\textcolor[rgb]{0.56,0.35,0.01}{\textbf{\textit{#1}}}}
\newcommand{\ErrorTok}[1]{\textcolor[rgb]{0.64,0.00,0.00}{\textbf{#1}}}
\newcommand{\ExtensionTok}[1]{#1}
\newcommand{\FloatTok}[1]{\textcolor[rgb]{0.00,0.00,0.81}{#1}}
\newcommand{\FunctionTok}[1]{\textcolor[rgb]{0.13,0.29,0.53}{\textbf{#1}}}
\newcommand{\ImportTok}[1]{#1}
\newcommand{\InformationTok}[1]{\textcolor[rgb]{0.56,0.35,0.01}{\textbf{\textit{#1}}}}
\newcommand{\KeywordTok}[1]{\textcolor[rgb]{0.13,0.29,0.53}{\textbf{#1}}}
\newcommand{\NormalTok}[1]{#1}
\newcommand{\OperatorTok}[1]{\textcolor[rgb]{0.81,0.36,0.00}{\textbf{#1}}}
\newcommand{\OtherTok}[1]{\textcolor[rgb]{0.56,0.35,0.01}{#1}}
\newcommand{\PreprocessorTok}[1]{\textcolor[rgb]{0.56,0.35,0.01}{\textit{#1}}}
\newcommand{\RegionMarkerTok}[1]{#1}
\newcommand{\SpecialCharTok}[1]{\textcolor[rgb]{0.81,0.36,0.00}{\textbf{#1}}}
\newcommand{\SpecialStringTok}[1]{\textcolor[rgb]{0.31,0.60,0.02}{#1}}
\newcommand{\StringTok}[1]{\textcolor[rgb]{0.31,0.60,0.02}{#1}}
\newcommand{\VariableTok}[1]{\textcolor[rgb]{0.00,0.00,0.00}{#1}}
\newcommand{\VerbatimStringTok}[1]{\textcolor[rgb]{0.31,0.60,0.02}{#1}}
\newcommand{\WarningTok}[1]{\textcolor[rgb]{0.56,0.35,0.01}{\textbf{\textit{#1}}}}
\usepackage{longtable,booktabs,array}
\usepackage{calc} % for calculating minipage widths
% Correct order of tables after \paragraph or \subparagraph
\usepackage{etoolbox}
\makeatletter
\patchcmd\longtable{\par}{\if@noskipsec\mbox{}\fi\par}{}{}
\makeatother
% Allow footnotes in longtable head/foot
\IfFileExists{footnotehyper.sty}{\usepackage{footnotehyper}}{\usepackage{footnote}}
\makesavenoteenv{longtable}
\usepackage{graphicx}
\makeatletter
\newsavebox\pandoc@box
\newcommand*\pandocbounded[1]{% scales image to fit in text height/width
  \sbox\pandoc@box{#1}%
  \Gscale@div\@tempa{\textheight}{\dimexpr\ht\pandoc@box+\dp\pandoc@box\relax}%
  \Gscale@div\@tempb{\linewidth}{\wd\pandoc@box}%
  \ifdim\@tempb\p@<\@tempa\p@\let\@tempa\@tempb\fi% select the smaller of both
  \ifdim\@tempa\p@<\p@\scalebox{\@tempa}{\usebox\pandoc@box}%
  \else\usebox{\pandoc@box}%
  \fi%
}
% Set default figure placement to htbp
\def\fps@figure{htbp}
\makeatother
\setlength{\emergencystretch}{3em} % prevent overfull lines
\providecommand{\tightlist}{%
  \setlength{\itemsep}{0pt}\setlength{\parskip}{0pt}}
\setcounter{secnumdepth}{-\maxdimen} % remove section numbering
% definitions for citeproc citations
\NewDocumentCommand\citeproctext{}{}
\NewDocumentCommand\citeproc{mm}{%
  \begingroup\def\citeproctext{#2}\cite{#1}\endgroup}
\makeatletter
 % allow citations to break across lines
 \let\@cite@ofmt\@firstofone
 % avoid brackets around text for \cite:
 \def\@biblabel#1{}
 \def\@cite#1#2{{#1\if@tempswa , #2\fi}}
\makeatother
\newlength{\cslhangindent}
\setlength{\cslhangindent}{1.5em}
\newlength{\csllabelwidth}
\setlength{\csllabelwidth}{3em}
\newenvironment{CSLReferences}[2] % #1 hanging-indent, #2 entry-spacing
 {\begin{list}{}{%
  \setlength{\itemindent}{0pt}
  \setlength{\leftmargin}{0pt}
  \setlength{\parsep}{0pt}
  % turn on hanging indent if param 1 is 1
  \ifodd #1
   \setlength{\leftmargin}{\cslhangindent}
   \setlength{\itemindent}{-1\cslhangindent}
  \fi
  % set entry spacing
  \setlength{\itemsep}{#2\baselineskip}}}
 {\end{list}}
\usepackage{calc}
\newcommand{\CSLBlock}[1]{\hfill\break\parbox[t]{\linewidth}{\strut\ignorespaces#1\strut}}
\newcommand{\CSLLeftMargin}[1]{\parbox[t]{\csllabelwidth}{\strut#1\strut}}
\newcommand{\CSLRightInline}[1]{\parbox[t]{\linewidth - \csllabelwidth}{\strut#1\strut}}
\newcommand{\CSLIndent}[1]{\hspace{\cslhangindent}#1}
\usepackage{fancyhdr}
\pagestyle{fancy}
\fancyhf{}
\lfoot[\thepage]{}
\rfoot[]{\thepage}
\fontsize{12}{22}
\selectfont
\usepackage{booktabs}
\usepackage{longtable}
\usepackage{array}
\usepackage{multirow}
\usepackage{wrapfig}
\usepackage{float}
\usepackage{colortbl}
\usepackage{pdflscape}
\usepackage{tabu}
\usepackage{threeparttable}
\usepackage{threeparttablex}
\usepackage[normalem]{ulem}
\usepackage{makecell}
\usepackage{xcolor}
\usepackage{bookmark}
\IfFileExists{xurl.sty}{\usepackage{xurl}}{} % add URL line breaks if available
\urlstyle{same}
\hypersetup{
  colorlinks=true,
  linkcolor={blue},
  filecolor={Maroon},
  citecolor={Blue},
  urlcolor={Blue},
  pdfcreator={LaTeX via pandoc}}

\title{\includegraphics[width=6cm,height=\textheight,keepaspectratio]{figures/IEO-logo.jpg}}
\author{}
\date{\vspace{-2.5em}}

\begin{document}
\maketitle


\pagenumbering{gobble}

%\begin{titlepage}
\begin{flushleft}
\Large{\textbf{Reporte Preliminar 1}}\\
\vspace*{2\baselineskip}
\LARGE{\textbf{Análisis del viento de Levante y su impacto en reclutamiento de coquina \textit{Donax trunculus} durante la última década}}\\
\vspace*{5\baselineskip}
\Large{Project FEMP 04}\\
\vspace*{1\baselineskip}
\Large{Instituto Español de Oceanografía, Cádiz }\\
\vspace*{4\baselineskip}
\end{flushleft}
\begin{flushright}
\large{\textit{Mauricio Mardones}}\\
\large{\textit{Marina Delgado}}\\
\large{\textit{Ricardo Sánchez}}\\
\large{\textit{Luis Silva}}\\
\vspace*{1\baselineskip}
\normalsize{\textbf{Fecha}}\\
August, 2025
\end{flushright}

% \end{titlepage}


\hypersetup{linkcolor = black}
\newpage
\pagenumbering{roman}
%\tableofcontents
%\addcontentsline{toc}{section}{\contentsname}

\newpage



\pagenumbering{arabic}
\hypersetup{linkcolor = blue}

{
\hypersetup{linkcolor=}
\setcounter{tocdepth}{3}
\tableofcontents
}
\pagebreak

\section{Contexto}\label{contexto}

Análisis de los vientos de Levante y su relación con el reclutamiento de la coquina (\emph{Donax trunculus}). Este estudio realiza un análisis detallado de los patrones de viento de Levante y Poniente a partir de datos meteorológicos recopilados entre 2013 y 2025. Se caracterizan aspectos clave como la frecuencia, duración e intensidad de estos eventos atmosféricos, con el objetivo de evaluar su posible influencia sobre variables poblacionales de la coquina \emph{Donax trunculus}, una especie bivalva de importancia ecológica y pesquera en el litoral suratlántico español.

El objetivo principal de este trabajo es identificar y cuantificar los patrones de viento dominantes y analizar su relación con parámetros poblacionales de \emph{Donax trunculus}. Para ello, se emplean datos de monitoreo biológico y pesquero obtenidos por el Instituto Español de Oceanografía (IEO-CSIC) en el marco del proyecto FEMP 04, con énfasis en las dinámicas de reclutamiento observadas en el Golfo de Cádiz.

\newpage

\section{Metodología}\label{metodologuxeda}

\begin{Shaded}
\begin{Highlighting}[]
\NormalTok{paquetes }\OtherTok{\textless{}{-}} \FunctionTok{c}\NormalTok{(}
  \StringTok{"readr"}\NormalTok{, }\StringTok{"dplyr"}\NormalTok{, }\StringTok{"lubridate"}\NormalTok{, }\StringTok{"stringr"}\NormalTok{, }\StringTok{"purrr"}\NormalTok{,}
  \StringTok{"ggplot2"}\NormalTok{, }\StringTok{"tidyr"}\NormalTok{, }\StringTok{"gridExtra"}\NormalTok{, }\StringTok{"viridis"}\NormalTok{, }\StringTok{"scales"}\NormalTok{,}
  \StringTok{"formatR"}\NormalTok{, }\StringTok{"ggpubr"}\NormalTok{, }\StringTok{"ggthemes"}\NormalTok{, }\StringTok{"kableExtra"}\NormalTok{, }\StringTok{"sjPlot"}\NormalTok{, }
  \StringTok{"broom"}\NormalTok{, }\StringTok{"kableExtra"}\NormalTok{, }\StringTok{"zoo"}\NormalTok{, }\StringTok{"tseries"}\NormalTok{, }\StringTok{"forecast"}\NormalTok{,}
  \StringTok{"seasonal"}\NormalTok{, }\StringTok{"plotly"}
\NormalTok{  )}

\NormalTok{purrr}\SpecialCharTok{::}\FunctionTok{walk}\NormalTok{(paquetes, library, }\AttributeTok{character.only =} \ConstantTok{TRUE}\NormalTok{)}
\end{Highlighting}
\end{Shaded}

\subsubsection{Datos de viento}\label{datos-de-viento}

El acceso al servicio de descargas de \textbf{Puertos del Estado} se hace desde la página web de oceanografía de \href{http://www.puertos.es/es-es/oceanografia/Paginas/portus.aspx}{Puertos del Estado}

Selección de estaciones y variables

\begin{itemize}
\tightlist
\item
  En el sistema \textbf{DescargaPortus}, se suele seleccionar entre varias estaciones meteorológicas costeras.
\item
  Elegimos \textbf{cuatro puntos} ubicados frente a la costa del Parque de Doñana en la provincia de Cádiz (Figure \ref{fig:mapa}).
\item
  Seleccionamos datos de \textbf{viento}, en particular \textbf{velocidad} y \textbf{dirección}.
\end{itemize}

\begin{figure}

{\centering \includegraphics[width=0.95\linewidth]{figures/mapa} 

}

\caption{Puntos seleccionados con la variable viento desde Puertos del Estado}\label{fig:mapa}
\end{figure}

Período y frecuencia de datos

\begin{itemize}
\tightlist
\item
  Un \textbf{rango temporal} (por ejemplo 2013‑2025).
\item
  Desechar valores nulos (ej. ‑999.9) siguiendo las recomendaciones de limpieza del manual.
\end{itemize}

Formato de descarga

\begin{itemize}
\tightlist
\item
  \texttt{.csv} en formato tabular con columnas como:
\item
  Fecha (GMT)
\item
  Velocidad del viento (m/s)
\item
  Dirección del viento (grados meteorológicos)
\end{itemize}

Calidad de datos

\begin{itemize}
\tightlist
\item
  Aplicamos filtros para descartar datos erróneos, como aquellos con velocidad negativa o direcciones fuera del rango 0--360º.
\end{itemize}

Procesamiento en R

\begin{itemize}
\tightlist
\item
  Finalmente, se importan en R usando funciones tipo \texttt{readr}, limpieza con \texttt{dplyr}, conversión de fechas con \texttt{lubridate}, etc.
\end{itemize}

Flujo de trabajo

\begin{longtable}[]{@{}
  >{\raggedright\arraybackslash}p{(\linewidth - 2\tabcolsep) * \real{0.3375}}
  >{\raggedright\arraybackslash}p{(\linewidth - 2\tabcolsep) * \real{0.6625}}@{}}
\toprule\noalign{}
\begin{minipage}[b]{\linewidth}\raggedright
Etapa
\end{minipage} & \begin{minipage}[b]{\linewidth}\raggedright
Acción realizada
\end{minipage} \\
\midrule\noalign{}
\endhead
\bottomrule\noalign{}
\endlastfoot
Selección de estaciones & Cuatro puntos frente a Doñana (Cádiz) \\
Variables & Viento: velocidad y dirección \\
Periodo de interés & Desde 2013 hasta el presente \\
Formato de descarga & CSV / delimitado con primer registro meta \\
Limpieza de datos & Eliminación de ``‑999.9'' y NA \\
Conversión de formatos & Fecha GMT a \texttt{POSIXct} o \texttt{Date} \\
\end{longtable}

\begin{Shaded}
\begin{Highlighting}[]
\NormalTok{directorio }\OtherTok{\textless{}{-}} \StringTok{"\textasciitilde{}/IEO/wind{-}recruitment{-}Dtrunculus/data"}
\NormalTok{archivos }\OtherTok{\textless{}{-}} \FunctionTok{c}\NormalTok{(}
  \StringTok{"21405\_40638\_5028023\_WIND\_20130101124124\_20250723114124.csv"}\NormalTok{,}
  \StringTok{"21405\_40639\_5027023\_WIND\_20130101124130\_20250723114130.csv"}\NormalTok{,}
  \StringTok{"21405\_40640\_5029023\_WIND\_20130101124134\_20250723114134.csv"}\NormalTok{,}
  \StringTok{"21405\_40641\_5030023\_WIND\_20130101124138\_20250723114138.csv"}
\NormalTok{)}
\NormalTok{rutas\_completas }\OtherTok{\textless{}{-}} \FunctionTok{file.path}\NormalTok{(directorio, archivos)}

\CommentTok{\#Funcion para leer todos los archivos.}
\NormalTok{leer\_archivo\_viento }\OtherTok{\textless{}{-}} \ControlFlowTok{function}\NormalTok{(archivo) \{}
\NormalTok{  readr}\SpecialCharTok{::}\FunctionTok{read\_tsv}\NormalTok{(archivo, }\AttributeTok{skip =} \DecValTok{1}\NormalTok{,}
                  \AttributeTok{col\_names =} \FunctionTok{c}\NormalTok{(}\StringTok{"fecha\_raw"}\NormalTok{, }
                                \StringTok{"velocidad\_viento"}\NormalTok{, }
                                \StringTok{"direccion\_grados"}\NormalTok{),}
                  \AttributeTok{show\_col\_types =} \ConstantTok{FALSE}\NormalTok{) }\SpecialCharTok{\%\textgreater{}\%}
    \FunctionTok{mutate}\NormalTok{(}
      \AttributeTok{fecha\_raw =}\NormalTok{ stringr}\SpecialCharTok{::}\FunctionTok{str\_trim}\NormalTok{(fecha\_raw),}
      \AttributeTok{fecha =}\NormalTok{ lubridate}\SpecialCharTok{::}\FunctionTok{parse\_date\_time}\NormalTok{(fecha\_raw, }\AttributeTok{orders =} \StringTok{"Y m d H"}\NormalTok{, }\AttributeTok{tz =} \StringTok{"UTC"}\NormalTok{),}
      \AttributeTok{velocidad\_viento =} \FunctionTok{as.numeric}\NormalTok{(velocidad\_viento),}
      \AttributeTok{direccion\_grados =} \FunctionTok{as.numeric}\NormalTok{(direccion\_grados)}
\NormalTok{    ) }\SpecialCharTok{\%\textgreater{}\%}
\NormalTok{    dplyr}\SpecialCharTok{::}\FunctionTok{select}\NormalTok{(fecha, velocidad\_viento, direccion\_grados)}
\NormalTok{\}}

\NormalTok{datos\_viento }\OtherTok{\textless{}{-}}\NormalTok{ purrr}\SpecialCharTok{::}\FunctionTok{map\_dfr}\NormalTok{(rutas\_completas, }
\NormalTok{                               leer\_archivo\_viento)}
\end{Highlighting}
\end{Shaded}

Los registros de viento fueron clasificados en tres categorías: \textbf{Levante}, \textbf{Poniente} u \textbf{Otro}, en base a la siguiente regla propuesta por Bartolomé López-Somoza (\citeproc{ref-bartolome1998viento}{1998}):

\[
\text{tipo\_viento} =
\begin{cases}
\text{"Levante"} & \text{si } 4 \leq V < 40 \text{ y } 67.5^\circ \leq D \leq 157.5^\circ \\
\text{"Poniente"} & \text{si } 8 \leq V < 40 \text{ y } 247.5^\circ \leq D \leq 292.5^\circ \\
\text{"Otro"} & \text{en cualquier otro caso}
\end{cases}
\]
Donde:

\begin{itemize}
\tightlist
\item
  \(V\) es la \textbf{velocidad del viento} en m/s.
\item
  \(D\) es la \textbf{dirección del viento} en grados meteorológicos.
\end{itemize}

Esta clasificación se implementó en R con \texttt{case\_when()} para etiquetar automáticamente cada observación horaria.

\begin{Shaded}
\begin{Highlighting}[]
\CommentTok{\# Clasificar los datos de viento}
\NormalTok{datos\_clasificados }\OtherTok{\textless{}{-}}\NormalTok{ datos\_limpios }\SpecialCharTok{\%\textgreater{}\%}
  \FunctionTok{mutate}\NormalTok{(}
    \AttributeTok{tipo\_viento =} \FunctionTok{case\_when}\NormalTok{(}
\NormalTok{      velocidad\_viento }\SpecialCharTok{\textgreater{}=} \DecValTok{4} \SpecialCharTok{\&}\NormalTok{ velocidad\_viento }\SpecialCharTok{\textless{}=} \DecValTok{40} \SpecialCharTok{\&}
\NormalTok{        direccion\_grados }\SpecialCharTok{\textgreater{}=} \FloatTok{67.5} \SpecialCharTok{\&}\NormalTok{ direccion\_grados }\SpecialCharTok{\textless{}=} \FloatTok{157.5} \SpecialCharTok{\textasciitilde{}} \StringTok{"Levante"}\NormalTok{,}
\NormalTok{      velocidad\_viento }\SpecialCharTok{\textgreater{}=} \DecValTok{8} \SpecialCharTok{\&}\NormalTok{ velocidad\_viento }\SpecialCharTok{\textless{}=} \DecValTok{40} \SpecialCharTok{\&}
\NormalTok{        direccion\_grados }\SpecialCharTok{\textgreater{}=} \FloatTok{247.5} \SpecialCharTok{\&}\NormalTok{ direccion\_grados }\SpecialCharTok{\textless{}=} \FloatTok{292.5} \SpecialCharTok{\textasciitilde{}} \StringTok{"Poniente"}\NormalTok{,}
      \ConstantTok{TRUE} \SpecialCharTok{\textasciitilde{}} \StringTok{"Otro"}
\NormalTok{    )}
\NormalTok{  )}
\end{Highlighting}
\end{Shaded}

En terminos direccionales, la rosa de los vientos queda clasificada de la siguiente forma (Figura \ref{fig:windrose}):

\begin{figure}

{\centering \includegraphics[width=1\linewidth]{Levante_Analisys_files/figure-latex/windrose-1} 

}

\caption{Rosa de vientos mostrando la clasificación de sectores para Levante y Poniente}\label{fig:windrose}
\end{figure}

\subsection{Datos de variables poblacionales de coquina}\label{datos-de-variables-poblacionales-de-coquina}

Utilizaremos el indice de reclutamiento que provien e de las tallas , calculado como la proporcion de individuos \texttt{\textless{}\ 15\ mm} los cuales se obtienen de las estructiras de tallas de la fraccion ``Poblacional'' como lo inica la Figura \ref{fig:recl}.

\begin{figure}

{\centering \includegraphics[width=0.99\linewidth]{figures/Recruit_2013_2024} 

}

\caption{Indice de Reclutamiento por año y por sector de muestreo}\label{fig:recl}
\end{figure}

\newpage

\section{Resultados}\label{resultados}

Analizamos la frecuencia de eventos de levante y poniente en la Figura \ref{fig:frequency-analysis2}.

\begin{figure}

{\centering \includegraphics[width=0.6\linewidth]{Levante_Analisys_files/figure-latex/frequency-analysis2-1} 

}

\caption{Distribución de velocidades del viento y frecuencia por dirección cardinal}\label{fig:frequency-analysis2}
\end{figure}

\newpage

También visualizamos la relación entre intensisdad y dias con levante y poniente en la Figura \ref{fig:intvel}

\begin{figure}

{\centering \includegraphics[width=0.6\linewidth]{Levante_Analisys_files/figure-latex/intvel-1} 

}

\caption{Relación entre intensisdad y dias con levante y poniente}\label{fig:intvel}
\end{figure}

Promedio por mes de los días con levante y poniente a través de los meses en la Figura \ref{fig:intvel2}.

\begin{figure}

{\centering \includegraphics[width=0.6\linewidth]{Levante_Analisys_files/figure-latex/intvel2-1} 

}

\caption{Relación entre intensisdad y dias con levante y poniente}\label{fig:intvel2}
\end{figure}

(Estas figuras son solo vizualizaciones dado que no son muy informativas)

\newpage

Ahora por año y por mes. La Figura \ref{fig:estaci} se muestra los eventos de Levante y Poniente por mes y por año

\begin{figure}

{\centering \includegraphics[width=1\linewidth]{Levante_Analisys_files/figure-latex/estaci-1} 

}

\caption{Detalle de vientos de levante y Poniente}\label{fig:estaci}
\end{figure}

La Figura \ref{fig:estaci2} muestra un mapa de calor de ambos vientos.

\begin{figure}

{\centering \includegraphics[width=1\linewidth]{Levante_Analisys_files/figure-latex/estaci2-1} 

}

\caption{Mapa de calor de vientos de levante y Poniente}\label{fig:estaci2}
\end{figure}

La Figura \ref{fig:estaciones} muestra la tendiencia temporal de ambos vientos.

\begin{figure}

{\centering \includegraphics[width=1\linewidth]{Levante_Analisys_files/figure-latex/estaciones-1} 

}

\caption{Serie temporal de vientos de levante y Poniente}\label{fig:estaciones}
\end{figure}

El fenomeno de Levante en el Parque Doñana muestra un comportamiento estacioal marcado.

\newpage

\subsubsection{Series temporales del monitoreo poblacional de coquina}\label{series-temporales-del-monitoreo-poblacional-de-coquina}

El reclutamiento a traves de los meses y años puede verse en la Figura \ref{fig:recluserie}

\begin{figure}

{\centering \includegraphics[width=1\linewidth]{Levante_Analisys_files/figure-latex/recluserie-1} 

}

\caption{Serie temporal de reclutaamiento}\label{fig:recluserie}
\end{figure}

El plot muestra una marcada estacionalidad de los reclutamientos, con un pulso posterior a los meses estivales como lo indica Delgado \& Silva (\citeproc{ref-Delgado2018}{2018}) en la Figura \ref{fig:ciclo}.

\begin{figure}

{\centering \includegraphics[width=0.75\linewidth]{figures/ciclo} 

}

\caption{Schematic representation of the reproductive cycle, periods of emission of gametes and related recruitment events in populations of D. trunculus from SW Spain. Black symbols represent the C1 cohort (from February-March) and grey symbols represent the C2 cohort (from July)}\label{fig:ciclo}
\end{figure}
\newpage

\subsection{Modelos de correlacion entre Reclutamiento y Viento de Levante}\label{modelos-de-correlacion-entre-reclutamiento-y-viento-de-levante}

Uno las bases que estan en los objetos \texttt{serie\_levante} y \texttt{datos} y vizualizo el comportamiento de la variable en la Figura \ref{fig:hist1}.

\begin{figure}

{\centering \includegraphics[width=0.55\linewidth]{Levante_Analisys_files/figure-latex/hist1-1} 

}

\caption{Distribución de variablees objetivo para el análisis de correlación}\label{fig:hist1}
\end{figure}

A su vez, miro como se distribuyen en el tiempo en la Figura \ref{fig:serie2}.

\begin{figure}

{\centering \includegraphics[width=1\linewidth]{Levante_Analisys_files/figure-latex/serie2-1} 

}

\caption{Serie de tiempo de ambas variables}\label{fig:serie2}
\end{figure}

\begin{table}[!h]
\centering
\caption{\label{tab:unnamed-chunk-5}Modelo lineal simple de reclutamiento de coquina y viento de levante}
\centering
\begin{tabular}[t]{l|r|r|r|r|r|r}
\hline
Término & Coeficiente & Error estándar & Estadístico t & Valor p & IC inferior 95\% & IC superior 95\%\\
\hline
\cellcolor{gray!10}{(Intercept)} & \cellcolor{gray!10}{24.657} & \cellcolor{gray!10}{2.644} & \cellcolor{gray!10}{9.323810} & \cellcolor{gray!10}{0.000} & \cellcolor{gray!10}{19.413} & \cellcolor{gray!10}{29.901}\\
\hline
dias\_con\_evento & -0.418 & 0.277 & -1.508633 & 0.134 & -0.967 & 0.131\\
\hline
\end{tabular}
\end{table}

En primera instancia, no se observa una correlación lineal significativa entre el viento de levante y el reclutamiento, con un coeficiente de determinación muy bajo (R2 = 0.021), indicando escaso poder explicativo.

Se procede a analizar la relación entre estas variables en una escala temporal diferente, específicamente a nivel semestral, evaluando la posible influencia del viento de levante sobre el reclutamiento de coquina.

\begin{table}[!h]
\centering
\begin{tabular}[t]{r|r|r}
\hline
semestre & correlacion & n\_semestres\\
\hline
\cellcolor{gray!10}{1} & \cellcolor{gray!10}{-0.1966243} & \cellcolor{gray!10}{10}\\
\hline
2 & 0.5225889 & 10\\
\hline
\end{tabular}
\end{table}

\newpage

\subsubsection{Modelos semestrales con interacción}\label{modelos-semestrales-con-interacciuxf3n}

\begin{itemize}
\tightlist
\item
  Modelo lineal semestral
\end{itemize}

\[
\text{Reclutamiento}_i = \beta_0 + \beta_1 \cdot \text{Viento}_i + \varepsilon_i
\]

Donde:

\begin{itemize}
\item
  \(\text{Reclutamiento}_i\): reclutamiento promedio en el semestre \(i\)
\item
  \(\text{Viento}_i\): número de días de viento en el semestre \(i\)
\item
  \(\beta_0\): intercepto
\item
  \(\beta_1\): efecto lineal de los días de viento
\item
  \(\varepsilon_i \sim \mathcal{N}(0, \sigma^2)\): error aleatorio
\item
  Modelo cuadrático semestral
\end{itemize}

\[
\text{Reclutamiento}_i = \beta_0 + \beta_1 \cdot \text{Viento}_i + \beta_2 \cdot \text{Viento}_i^2 + \varepsilon_i
\]

Donde:

\begin{itemize}
\item
  \(\beta_2\): efecto cuadrático de los días de viento
\item
  Modelo con efecto del semestre
\end{itemize}

\[
\text{Reclutamiento}_i = \beta_0 + \beta_1 \cdot \text{Viento}_i + \gamma_j \cdot \text{Semestre}_i + \varepsilon_i
\]

Donde:

\begin{itemize}
\item
  \(\text{Semestre}_i\): variable categórica que indica si el semestre es 1 o 2
\item
  \(\gamma_j\): efecto fijo del semestre (dummy variable, usualmente para semestre 2)
\item
  Modelo con interacción viento × semestre
\end{itemize}

\[
\text{Reclutamiento}_i = \beta_0 + \beta_1 \cdot \text{Viento}_i + \gamma_j \cdot \text{Semestre}_i + \delta_j \cdot (\text{Viento}_i \cdot \text{Semestre}_i) + \varepsilon_i
\]

Donde:

\begin{itemize}
\tightlist
\item
  \(\delta_j\): efecto de interacción entre el viento y el semestre
\end{itemize}

\begin{table}[!h]
\centering
\begin{tabular}[t]{l|l|r|r|r}
\hline
  & Modelo & R2\_ajustado & AIC & p\_value\\
\hline
\cellcolor{gray!10}{Lineal} & \cellcolor{gray!10}{Lineal} & \cellcolor{gray!10}{-0.0554818} & \cellcolor{gray!10}{159.4193} & \cellcolor{gray!10}{0.9720995}\\
\hline
Cuadrático & Cuadrático & -0.0469758 & 160.1143 & 0.5739469\\
\hline
\cellcolor{gray!10}{Con semestre} & \cellcolor{gray!10}{Con semestre} & \cellcolor{gray!10}{0.3456430} & \cellcolor{gray!10}{150.7142} & \cellcolor{gray!10}{0.0105642}\\
\hline
Interacción & Interacción & 0.4262550 & 148.8723 & 0.0074818\\
\hline
\end{tabular}
\end{table}

\newpage

Figura \ref{fig:resultlm} vemos los resultados de la prediccion con el modelo seleccionado \texttt{Modelo\ con\ interacción\ viento\ ×\ semestre}

\begin{figure}[ht!]

{\centering \includegraphics[width=1\linewidth]{Levante_Analisys_files/figure-latex/resultlm-1} 

}

\caption{Cada punto = un semestre (n=20). Modelo: Reclutamiento ~ Viento × Semestre}\label{fig:resultlm}
\end{figure}

Efectos opuestos del viento de Levante según el período del año. El gráfico muestra que la influencia del viento de Levante sobre el reclutamiento de coquina (\emph{Donax trunculus}) varía significativamente según el semestre del año, revelando una interacción estacional. Durante el primer semestre (enero a junio), que corresponde al periodo reproductivo, se observa un efecto negativo, es decir, a mayor número de días con viento de Levante, menor es el reclutamiento promedio. Esto sugiere que estos vientos podrían interferir con el éxito reproductivo, afectando la dispersión larval o la retención costera, o generando condiciones ambientales desfavorables en el hábitat de asentamiento, como cambios en temperatura, salinidad o estructura del sedimento. En cambio, en el segundo semestre (julio a diciembre), el efecto es positivo: un mayor número de días con viento se asocia con un mayor reclutamiento, lo que podría deberse a que estos vientos favorecen la acumulación de materia orgánica o sedimentos en las playas, mejorando las condiciones de crecimiento y refugio para los juveniles ya asentados. Esta inversión del efecto sugiere que el mismo factor ambiental puede tener consecuencias opuestas según el momento del ciclo de vida en que actúa, lo que subraya la importancia de considerar la estacionalidad en la evaluación del reclutamiento y en la gestión pesquera de esta especie.

\newpage

\subsubsection{Series de Tiempo}\label{series-de-tiempo}

Crear métricas anuales del viento Levante. Como tengo registros por horas vacios, saco la variable de forma quincenal (Figura \ref{fig:seretem}).

\begin{figure}

{\centering \includegraphics[width=1\linewidth]{Levante_Analisys_files/figure-latex/seretem-1} 

}

\caption{Serie temporal quincenal de viento de Levante}\label{fig:seretem}
\end{figure}

Ahora veo el comprtamiento anual dd la variable

\textbf{Velocidad media del viento anual}

Es el promedio de las velocidades diarias dentro de un año:
\[
\bar{v} = \frac{1}{n} \sum_{i=1}^{n} v_i
\]

Donde \(v_i\) es la velocidad del viento el día \(i\) y \(n\) es el número de días del año.

\textbf{Velocidad máxima del viento anual}
Valor máximo observado en el año:

\[
v_{\text{max}} = \max(v_1, v_2, \dots, v_n)
\]
\textbf{Velocidad mínima del viento anual}
Valor mínimo observado en el año:

\[
v_{\text{min}} = \min(v_1, v_2, \dots, v_n)
\]

\textbf{Percentil 95 (P95)}
Valor por debajo del cual cae el 95\% de las observaciones:

\[
P_{95} = \text{percentil}_{95}(v_1, v_2, \dots, v_n)
\]

\textbf{Percentil 75 (P75)}
Valor por debajo del cual cae el 75\% de las observaciones:

\[
P_{75} = \text{percentil}_{75}(v_1, v_2, \dots, v_n)
\]

\textbf{Percentil 25 (P25)}
Valor por debajo del cual cae el 25\% de las observaciones:

\[
P_{25} = \text{percentil}_{25}(v_1, v_2, \dots, v_n)
\]

\textbf{Días con viento fuerte (por encima del percentil 80 del año)}
Cuenta de días cuya velocidad supera el percentil 80:

\[
D_{80} = \sum_{i=1}^{n} \mathbb{1}(v_i > P_{80})
\]

Donde \(\mathbb{1}\) es la función indicadora y \(P_{80}\) es el percentil 80 de la serie del año.

\textbf{Variabilidad (desviación estándar)}
Medida de dispersión en torno a la media:

\[
\sigma = \sqrt{\frac{1}{n - 1} \sum_{i=1}^{n} (v_i - \bar{v})^2}
\]
\textbf{Número de días con datos válidos (n)}
Total de observaciones disponibles en el año:

\[
n = \text{número de días con datos}
\]

Ahora las graficas con estas metricas del levante como análisis exploratorio (Figura \ref{fig:levmet}).

\begin{figure}

{\centering \includegraphics[width=0.8\linewidth]{Levante_Analisys_files/figure-latex/levmet-1} 

}

\caption{Metricas anuales del viento de Levante}\label{fig:levmet}
\end{figure}

Ahora utilizo \texttt{ts} funcion para connvertir a serie temporaly visualizo la serie en la Figura \ref{fig:serie3}

\begin{Shaded}
\begin{Highlighting}[]
\CommentTok{\# 1. Convertir a serie temporal (24 quincenas por año)}
\NormalTok{ts\_levante }\OtherTok{\textless{}{-}} \FunctionTok{ts}\NormalTok{(datos\_levante}\SpecialCharTok{$}\NormalTok{velocidad\_promedio,}
                 \AttributeTok{start =} \FunctionTok{c}\NormalTok{(}\DecValTok{2013}\NormalTok{, }\DecValTok{1}\NormalTok{), }\AttributeTok{frequency =} \DecValTok{24}\NormalTok{)}
\end{Highlighting}
\end{Shaded}

\begin{figure}

{\centering \includegraphics[width=0.8\linewidth]{Levante_Analisys_files/figure-latex/serie3-1} 

}

\caption{Serie de tiempo del viento de Levante}\label{fig:serie3}
\end{figure}

Para explorar la estructura temporal de la ocurrencia del viento de Levante, se utilizaron las funciones \texttt{acf()} y \texttt{pacf()} en R, que permiten examinar la dependencia serial de la variable en el tiempo. La función \texttt{acf} (función de autocorrelación) evalúa cómo los valores de días con viento de Levante en un periodo se correlacionan con sus propios valores en rezagos anteriores, lo que permite identificar patrones de persistencia, repetición o estacionalidad. Complementariamente, la función \texttt{pacf} (función de autocorrelación parcial) estima la correlación entre la serie y sus rezagos eliminando el efecto de los rezagos intermedios, siendo útil para distinguir relaciones directas a distintos desfases. En el contexto del Levante, estas herramientas permiten detectar si existen periodos de ocurrencia agrupada (eventos persistentes) o ciclos que podrían tener implicancias ecológicas, por ejemplo, sobre procesos de reclutamiento o dispersión larval influenciados por este patrón de viento (Figura \ref{fig:autoc})

\begin{figure}

{\centering \includegraphics[width=0.8\linewidth]{Levante_Analisys_files/figure-latex/autoc-1} 

}

\caption{ACF y PACF para el viento de Levante}\label{fig:autoc}
\end{figure}

En el gráfico de ACF (Autocorrelación), se observa un valor inicial elevado en el \textbf{lag 1}, indicando una leve correlación entre la velocidad del viento de un periodo quincenal y el siguiente. Sin embargo, a partir del segundo desfase (lag 2), los valores de autocorrelación caen rápidamente y permanecen dentro del intervalo de confianza (líneas azules), lo que sugiere que \textbf{no existe una dependencia temporal sostenida}. Esto indica que la serie tiene \textbf{poca memoria}: el valor del viento en una quincena está apenas relacionado con el valor anterior y prácticamente independiente de los valores más lejanos en el tiempo.

En el gráfico de PACF (Autocorrelación Parcial), se observa un patrón disperso de barras que oscilan cerca del cero y dentro de los límites de significancia, sin un corte definido en los primeros lags. Esto sugiere la ausencia de una \textbf{estructura autorregresiva fuerte}, es decir, \textbf{los valores actuales no están directamente influidos por uno o dos valores anteriores de forma consistente}. Este patrón es típico de una serie dominada por ruido blanco o con estructura estocástica débil, lo que concuerda con el modelo ARIMA(0,0,1) ajustado previamente, que indica una dinámica simple con un componente de media móvil leve y escasa estructura autoregresiva o estacional.

\begin{verbatim}
## Test ADF p-value: 0.01
\end{verbatim}

\begin{verbatim}
## La serie ES estacionaria (p < 0.05)
\end{verbatim}

El test de estacionariedad Augmented Dickey-Fuller (ADF) aplicado a la serie temporal \texttt{ts\_levante} arrojó un p-valor de 0.01, que es menor al umbral comúnmente utilizado de 0.05. Esto indica que podemos rechazar la hipótesis nula de que la serie tiene una raíz unitaria (es decir, no es estacionaria). Por lo tanto, concluimos que la serie \texttt{ts\_levante} es estacionaria, lo que significa que sus propiedades estadísticas, como la media y la varianza, se mantienen constantes en el tiempo. Esto es fundamental para muchos análisis de series temporales y modelos predictivos que asumen estacionariedad.

\begin{verbatim}
## 
##  Jarque Bera Test
## 
## data:  as.numeric(ts_levante)
## X-squared = 101.96, df = 2, p-value < 2.2e-16
\end{verbatim}

El test de Jarque-Bera arroja un p-value \textless{} 2.2e-16, lo que indica un rechazo categórico de la hipótesis nula de normalidad para la serie temporal del viento Levante, evidenciando la presencia de asimetría y curtosis significativamente diferentes de una distribución gaussiana. Este resultado es típico en datos meteorológicos de viento, donde las distribuciones suelen presentar sesgo positivo debido a la alta frecuencia de valores bajos de velocidad y la presencia de eventos extremos ocasionales, además del truncamiento natural en cero que caracteriza las mediciones de velocidad del viento. Aunque esta desviación de la normalidad no compromete la validez del modelado ARIMA, que es robusto ante violaciones de este supuesto en la serie original, sí sugiere que los intervalos de confianza de los pronósticos pueden requerir métodos no paramétricos para mayor precisión, siendo más crítica la evaluación de normalidad en los residuos del modelo final que en la serie temporal original.

Descomponer la serie temporal del viento Levante es fundamental antes de realizar análisis de correlación con el reclutamiento porque la descomposición permite separar claramente las distintas componentes que conforman la serie: tendencia, estacionalidad y ruido (componente irregular) (Figura \ref{fig:descom})

Esto es importante por varias razones:

\textbf{Aislar la señal relevante:} La variabilidad estacional y las tendencias a largo plazo pueden enmascarar o distorsionar la relación real entre el viento Levante y el reclutamiento. Por ejemplo, una tendencia creciente en el viento o patrones estacionales fuertes podrían inducir correlaciones espurias si no se controlan.

\textbf{Analizar relaciones específicas:} Al separar la serie en componentes, podemos investigar si el reclutamiento está correlacionado con la tendencia (por ejemplo, cambios a largo plazo en el viento), con la estacionalidad (patrones periódicos) o con las fluctuaciones residuales (ruido). Esto ayuda a entender mejor qué aspectos del viento Levante influyen en el reclutamiento.

\textbf{Mejorar modelos predictivos:} Modelos que usan series descompuestas o componentes filtradas suelen ser más robustos y con mayor poder explicativo porque evitan la confusión causada por la mezcla de señales.

\textbf{Cumplir supuestos estadísticos:} Muchos métodos estadísticos, como la correlación o regresión, asumen que las series son estacionarias o que las variables no están afectadas por tendencias o estacionalidades no controladas.

Por eso, la descomposición STL (que es flexible y puede adaptarse a cambios en la estacionalidad) y la descomposición clásica permiten obtener una visión clara y precisa de la dinámica del viento Levante, facilitando así una interpretación más acertada de cómo este factor climático puede afectar el reclutamiento de la población estudiada.

\begin{figure}

{\centering \includegraphics[width=1\linewidth]{Levante_Analisys_files/figure-latex/descom-1} 

}

\caption{Descomposicion de la serie de tiempo del viento de Levante}\label{fig:descom}
\end{figure}

El gráfico de descomposición STL muestra la descomposición aditiva de la serie diaria de viento de Levante en sus tres componentes: tendencia, estacionalidad y residuo. En la primera franja se observa la serie original con alta variabilidad diaria. La componente estacional, en la segunda franja, revela un patrón repetitivo que parece reflejar cierta estacionalidad intraanual, posiblemente asociada a ciclos meteorológicos recurrentes, aunque con variabilidad de amplitud a lo largo del tiempo. Esto sugiere que hay cierta estructura estacional, aunque no muy pronunciada o perfectamente regular.

La tercera franja muestra la tendencia suavizada, la cual evidencia una leve pendiente ascendente entre 2013 y mediados de 2015, seguida por una caída y luego un aumento marcado hacia fines de 2015. Esto podría reflejar un cambio en la intensidad promedio del viento de Levante en ese periodo. Por último, el componente de residuo, en la cuarta franja, conserva una gran parte de la variabilidad diaria, lo que indica que buena parte de la señal es ruido o está influenciada por fenómenos no capturados ni por la estacionalidad ni por la tendencia, lo cual es esperable en datos meteorológicos con alta variabilidad a corto plazo.

Extraer componentes para usar en modelos posteriores

\subsubsection{Guardar datos para modelos posteriores}\label{guardar-datos-para-modelos-posteriores}

\begin{Shaded}
\begin{Highlighting}[]
\FunctionTok{write.csv}\NormalTok{(df\_componentes, }
          \StringTok{"levante\_componentes.csv"}\NormalTok{, }\AttributeTok{row.names =} \ConstantTok{FALSE}\NormalTok{)}
\end{Highlighting}
\end{Shaded}

\subsubsection{Prediccion (ARIMA) (En desarrollo\ldots)}\label{prediccion-arima-en-desarrollo}

\begin{center}\includegraphics[width=1\linewidth]{Levante_Analisys_files/figure-latex/unnamed-chunk-17-1} \end{center}

Visualizar media Movil

\begin{center}\includegraphics[width=1\linewidth]{Levante_Analisys_files/figure-latex/unnamed-chunk-18-1} \end{center}

Con el archivo \texttt{levante\_componentes.csv}, puedes:

\begin{itemize}
\tightlist
\item
  Hacer correlaciones con tus datos de reclutamiento (\texttt{lm(reclutamiento\ \textasciitilde{}\ tendencia)}).
\item
  Incluir \texttt{tendencia}, \texttt{estacionalidad} o incluso \texttt{residuo} como covariables en tus modelos lineales o no lineales.
\end{itemize}

\subsubsection{Trabajo futuro}\label{trabajo-futuro}

Incorporar otras variables predictoras como temperatura, clorofila, entre otras\ldots{}

\newpage

\section{Material Suplementario}\label{material-suplementario}

El codigo fuente para el análisis exploratorio y manipulación de datos de viento y reclutamiento de coquina se encuentran disponibles en el repositorio de GitHub: \href{https://github.com/MauroMardones/wind-recruitment-Dtrunculus}{wind\_recruit\_Dtrunculus}.

\newpage

\section*{Referencias}\label{referencias}
\addcontentsline{toc}{section}{Referencias}

\phantomsection\label{refs}
\begin{CSLReferences}{1}{0}
\bibitem[\citeproctext]{ref-bartolome1998viento}
Bartolomé López-Somoza, E. (1998). \emph{El viento de levante en el puerto}. Ayuntamiento de El Puerto de Santa María.

\bibitem[\citeproctext]{ref-Delgado2018}
Delgado, M., \& Silva, L. (2018). {Timing variations and effects of size on the reproductive output of the wedge clam Donax trunculus (L. 1758) in the littoral of Huelva (SW Spain)}. \emph{Journal of the Marine Biological Association of the United Kingdom}, \emph{98}(2), 341--350. \url{https://doi.org/10.1017/S0025315416001429}

\end{CSLReferences}

\end{document}
